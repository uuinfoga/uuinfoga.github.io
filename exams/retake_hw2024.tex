\documentclass{article}
\usepackage{xspace}
\usepackage{fancyhdr}
\usepackage{graphicx}
\usepackage{amsmath,amssymb,amsthm}
\usepackage{microtype}
\usepackage{enumitem}
\usepackage{ifthen}
\usepackage{totcount}
\usepackage[usenames]{xcolor}
\usepackage[most]{tcolorbox}
\usepackage[a4paper, margin=1in]{geometry}


\graphicspath{ {./figures/} }

%--------------------------------------------------------------------------------

\newcommand{\edition}{2024-2025}
\newcommand{\deadline}{19 March 2025, 23:59}
\newcommand{\hwNum}{Retake}

%--------------------------------------------------------------------------------
% Style stuff
\pagestyle{fancy}
\lhead{Geometric Algorithms (INFOGA \edition)}
\chead{}
\rhead{\mytitle}
\lfoot{INFOGA \edition}
\cfoot{}
\rfoot{\thepage}
\renewcommand{\headrulewidth}{0.4pt}
\renewcommand{\footrulewidth}{0.4pt}
\renewcommand{\familydefault}{\sfdefault}

%--------------------------------------------------------------------------------
% Question macro's

\newcommand{\withpoints}[1]{%
  \addtocounter{pointscounter}{#1} \printpoints{#1}
}
\newcommand{\printpoints}[1]{%
   \ifthenelse{#1 = 0}
              {}
              {\textit{(#1 points)}}\mbox{}
}



\newenvironment{question}[1][0]{\begin{tcolorbox}[parbox=false
                                      ,breakable=true
                                      ,enhanced jigsaw
                                      ,title=\bfseries Question
                                      \stepcounter{questionscounter}\arabic{questionscounter}
                                      \withpoints{#1}]
                         }
                         {\end{tcolorbox}}
\newenvironment{subquestions}{\begin{enumerate}[leftmargin=*
                                        ]}
                             {\end{enumerate}}
\newcommand{\subquestion}[1][0]{\item \withpoints{#1}}

\newtotcounter{questionscounter}
\newtotcounter{pointscounter}

\newcommand{\numquestions}{\total{questionscounter}}
\newcommand{\numpoints}{\total{pointscounter}}

%--------------------------------------------------------------------------------
\newcommand{\myremark}[3]{\textcolor{blue}{\textsc{#1 #2:}} \textcolor{red}{\textsf{#3}}}
% \renewcommand{\myremark}[3]{}
\newcommand{\frank}[2][says]{\myremark{Frank}{#1}{#2}}

%--------------------------------------------------------------------------------
% Theorem Environments
\newtheorem{theorem} {Theorem}
\newtheorem{lemma}[theorem] {Lemma}
\newtheorem{corollary}[theorem] {Corollary}
\newtheorem{problem}[theorem] {Problem}
\newtheorem{observation}[theorem] {Observation}
\newtheorem{claim}[theorem] {Claim}
\newtheorem{invariant}[theorem] {Invariant}

%--------------------------------------------------------------------------------
% Otherwise useful Macro's
\newcommand{\eps}{\ensuremath{\varepsilon}\xspace}
\DeclareMathOperator{\argmin}{argmin}

\newcommand{\mkmbb}[1]{\ensuremath{\mathbb{#1}}\xspace}
\newcommand{\R}{\mkmbb{R}}

%--------------------------------------------------------------------------------
% The Actual document

\newcommand{\mytitle}{Homework Exam \hwNum\xspace\edition}
\title{\mytitle}
\date{\textbf{Deadline: } \deadline}
\author{{\color{red} My name and StudentID go here!}}


\begin{document}
\maketitle
\thispagestyle{fancy}
\noindent
  This homework exam has \numquestions\xspace questions for a total of
  \numpoints\xspace points. You can earn 10 additional points by a
  careful preparation of your hand-in: using a good layout, good
  spelling, good figures, no sloppy notation, no statements like ``The
  algorithm runs in $n\log n$.''  (forgetting the $O(..)$ and
  forgetting to say that it concerns time), etc. Use lemmas, theorems,
  and figures where appropriate. Your final grade will be the number
  of points divided by 10. Unless stated otherwise, both randomized
  and deterministic solutions are allowed. In case you are asked to
  analyze the running time and your algorithm is randomized, analyze
  its expected running time.

  \begin{question}[10] Let $P$ be a set of $n$ points in $\R^3$, and let
    $h$ be a real number. You can assume that all coordinates are
    unique. Consider a horizontal slab $S$ of height $h$
    (i.e. $S=\{(x,y,z_{\min}+z') \mid x,y \in \R, z' \in [0,h]\}$ for
    some $z_{\min} \in \R$), and let $V(S)$ denote the volume of the
    axis-aligned bounding box of $P \cap S$. Develop an $O(n\log n)$
    time algorithm that can compute the maximum value of $V(S)$ over
    all horizontal slabs $S$ of height $h$. Argue that your algorithm
    is correct and achieves the desired running time.
  \end{question}

  \begin{question}
    Let $\mathcal{S}$ be a planar subdivision with $n$
    vertices, represented as a DCEL.

    \begin{subquestions}
      \subquestion[10] Give pseudo-code for an algorithm that, given a
      pointer to a face $f$, computes \emph{the set} of all faces of
      $\mathcal{S}$ adjacent to $f$. Your algorithm should use the
      'Twin', 'NextEdge', 'PrevEdge', etc. fields to navigate
      (i.e. you cannot assume you can directly access a list of
      vertices). Describe in a few sentences the main idea of your
      algorithm. Furthermore, argue that it is correct.

      \subquestion[5] Analyze the running time of your algorithm. In
      particular, argue whether or not it is output sensitive, i.e. if
      its running time depends on $k$ the number of faces reported.
    \end{subquestions}
  \end{question}

  \begin{question}
    A simple polygon $P$ is \emph{star-shaped} if and only if it
    contains a point $q \in P$ such that for any point $p \in P$, the line segment
    $\overline{pq}$ lies inside $P$.
    \begin{subquestions}
      \subquestion[10] We say that $P$ is \emph{strictly star-shaped} if (and only
      if) the point $q$ is also a vertex of $P$. Show that there are polygons
      that are star-shaped but not strictly star-shaped.

      \subquestion[10] Design an algorithm that can decide if a simple polygon $P$
      with $n$ vertices is star-shaped in $O(n)$ expected time. You can again
      assume no three vertices of $P$ are colinear.
    \end{subquestions}
  \end{question}

  \begin{question}[20]
    Let $S$ be a set of $n$ line segments in $\R^2$. You can assume no
    three endpoints in $S$ are colinear, and all coordinate values of
    all endpoints are unique. Describe an $O(n^2)$ time algorithm to
    compute the maximum number of segments from $S$ that can be
    stabbed by a line. Your algorithm should report this number, $k$,
    and a line $\ell$ realizing $k$ intersections.
  \end{question}

  \begin{question}
    Let $P$ be a polyline with $n$ vertices. You can assume that $P$
    does not have any self intersections, that all coordinates of the
    vertices are unique, and that no three vertices are colinear.

    \begin{subquestions}
      \subquestion[5] Describe a data structure that can efficiently test if
      a query point $q \in \R^2$ lies on an edge of $P$, and if so,
      returns this edge. In particular, your data structure should
      answer these queries in worst case $O(\log n)$ time. Briefly
      analyze the preprocessing and space required.

      \subquestion[7] Given two points $s,t \in \R^2$, let
      $P[s,t] \subseteq P$ be the maximal polyline with starting point
      $s$ and ending point $t$ (and note that $P[s,t] = \emptyset$ if
      $s$ or $t$ does not lie on $P$).

      Describe a data structure that stores $P$ and can answer the
      following queries in $O(\log^c n)$ (expected or worst case)
      time, for some $c > 1$: given two query points $s,t \in \R^2$
      report the axis-parallel bounding box of $P[s,t]$.

      Analyze the space usage of your data structure and its
      preprocessing time.

      The number of points awarded for this question depends on the
      query time and space usage of your data structure. Aim for the
      fastest possible queries, while using subquadratic space.

      \subquestion[10] Describe a data structure that stores $P$ and can
      answer the following queries in $O(\log^c n)$ (expected or worst
      case) time, for some $c > 1$: given three points
      $s,t,q \in \R^2$, report the vertex in $P[s,t]$ closest to
      $q$. Analyze the space usage of your data structure and its
      preprocessing time.

      The number of points awarded for this question depends on the
      query time and space usage of your data structure. Aim for the
      fastest possible queries, while using subquadratic space.

      \subquestion[3] Suppose that $P$ may have self intersections. Briefly
      describe why/where in your solution for question c you need that
      $P$ has no self-intersections.

    \end{subquestions}
  \end{question}
\end{document}
