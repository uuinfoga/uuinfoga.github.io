\documentclass{article}
\usepackage{xspace}
\usepackage{fancyhdr}
\usepackage{graphicx}
\usepackage{amsmath,amssymb,amsthm}
\usepackage{microtype}
\usepackage{enumitem}
\usepackage{ifthen}
\usepackage{totcount}
\usepackage{algpseudocode}
\usepackage[usenames]{xcolor}
\usepackage[most]{tcolorbox}
\usepackage[a4paper, margin=1in]{geometry}

\graphicspath{ {./figures/} }

%--------------------------------------------------------------------------------

\newcommand{\edition}{2024-2025}
\newcommand{\deadline}{15 January 2023, 13:15}
\newcommand{\hwNum}{3}

\newcommand{\student}{\color{red} My name and StudentID go here!}

%--------------------------------------------------------------------------------
% Style stuff
\pagestyle{fancy}
\lhead{Geometric Algorithms (INFOGA \edition)}
\chead{}
\rhead{\mytitle}
\lfoot{INFOGA \edition}
\cfoot{\student}
\rfoot{\thepage}
\renewcommand{\headrulewidth}{0.4pt}
\renewcommand{\footrulewidth}{0.4pt}
\renewcommand{\familydefault}{\sfdefault}

%--------------------------------------------------------------------------------
% Question macro's

\newcommand{\withpoints}[1]{%
  \addtocounter{pointscounter}{#1} \printpoints{#1}
}
\newcommand{\printpoints}[1]{%
   \ifthenelse{#1 = 0}
              {}
              {\textit{(#1 points)}}\mbox{}
}



\newenvironment{question}[1][0]{\begin{tcolorbox}[parbox=false
                                      ,breakable=true
                                      ,enhanced jigsaw
                                      ,title=\bfseries Question
                                      \stepcounter{questionscounter}\arabic{questionscounter}
                                      \withpoints{#1}]
                         }
                         {\end{tcolorbox}}
\newenvironment{subquestions}{\begin{enumerate}[leftmargin=*,label=\alph*.
                                        ]}
                             {\end{enumerate}}
\newcommand{\subquestion}[1][0]{\item \withpoints{#1}}

\newtotcounter{questionscounter}
\newtotcounter{pointscounter}

\newcommand{\numquestions}{\total{questionscounter}}
\newcommand{\numpoints}{\total{pointscounter}}

%--------------------------------------------------------------------------------
\newcommand{\myremark}[3]{\textcolor{blue}{\textsc{#1 #2:}} \textcolor{red}{\textsf{#3}}}
% \renewcommand{\myremark}[3]{}
\newcommand{\frank}[2][says]{\myremark{Frank}{#1}{#2}}

%--------------------------------------------------------------------------------
% Theorem Environments
\newtheorem{theorem} {Theorem}
\newtheorem{lemma}[theorem] {Lemma}
\newtheorem{corollary}[theorem] {Corollary}
\newtheorem{problem}[theorem] {Problem}
\newtheorem{observation}[theorem] {Observation}
\newtheorem{claim}[theorem] {Claim}
\newtheorem{invariant}[theorem] {Invariant}

%--------------------------------------------------------------------------------
% Otherwise useful Macro's
\newcommand{\eps}{\ensuremath{\varepsilon}\xspace}
\DeclareMathOperator{\argmin}{argmin}
\DeclareMathOperator{\argmax}{argmax}

\newcommand{\mkmbb}[1]{\ensuremath{\mathbb{#1}}\xspace}
\newcommand{\R}{\mkmbb{R}}
\DeclareMathOperator{\Vor}{Vor}

\newcommand{\mathfunc}[1]{\ensuremath{\mathit{#1}}\xspace}

%--------------------------------------------------------------------------------
% The Actual document

\newcommand{\mytitle}{Homework Exam \hwNum\xspace\edition}
\title{\mytitle}
\date{\textbf{Deadline: } \deadline}
\author{\student}


\begin{document}
\maketitle
\thispagestyle{fancy}
  \noindent
  This homework exam has \numquestions\xspace questions for a total of
  \numpoints\xspace points. You can earn 10 additional points by a
  careful preparation of your hand-in: using a good layout, good
  spelling, good figures, no sloppy notation, no statements like ``The
  algorithm runs in $n\log n$.''  (forgetting the $O(..)$ and
  forgetting to say that it concerns time), etc. Use lemmas, theorems,
  and figures where appropriate. Your final grade will be the number
  of points divided by 10. Unless stated otherwise, both randomized
  and deterministic solutions are allowed. In case you are asked to
  analyze the running time and your algorithm is randomized, analyze
  its expected running time.

  \begin{question}
    \begin{subquestions}
      \subquestion[10] Let $P$ be a set of $n$ points in $\R^2$. Prove
      that the average degree of any triangulation on $P$ is at most
      6.

      \subquestion[10] Prove that for any $n$ there exists a set $P$ of
      $n$ points in $\R^2$ so that any triangulation of $P$ has a
      vertex of degree $n-1$.
    \end{subquestions}
  \end{question}

  \begin{question}[15]
    Suppose that we have a trace $\mathcal{T}$ that records the $n$
    operations that an algorithm executes on a (dynamic) dictionary
    storing some dynamic set $X$ of real numbers. That is, every entry
    in $\mathcal{T}$ is a triple consisting of a time stamp $t$, an
    operation, which is either \textsc{Insert} or \textsc{Delete}, and
    a value $v \in \R$ --the number that is inserted or
    deleted. Describe a data structure that allows us to efficiently
    ``replay'' the queries of the algorithm. That is; given an
    arbitrary query pair $(t,q)$ consisting of a time $t \in \R$ and a
    value $q \in \R$ it allows us to efficiently report the value $v$
    that was the successor of $q$ at time $t$. Analyze the space,
    preprocessing time, and query time of your solution.

    The number of points awarded will depend on the space,
    preprocessing, and query time of your solution.
  \end{question}

  \newcommand{\Sqs}{\ensuremath{\mathcal{S}}\xspace}
  \begin{question}[10]
    Let $\Sqs$ be a set of $n$ axis-parallel squares. Develop a data
    structure that can store $\Sqs$ that can answer the following
    queries: given a point $q$, report a largest square $S^* \in \Sqs$
    that contains $q$. Argue/prove that your data structure answers
    queries correctly. Aim for the fastest queries possible, while
    using $O(n\log^c n)$ space (for some constant $c$).

    The number of points awarded will depend on the query time and the
    space used by your data structure.
  \end{question}

  \newcommand{\T}{\ensuremath{\mathcal{T}}\xspace}
  \begin{question}[10]
    Let \T be a kD-tree on a set $P$ of $n$ points in $\R^2$, in which
    each node has been annotated with the number of points in its
    subtree. Analyze the worst case query time for a range counting
    query on \T with a query disk $D$. Argue that your analysis is
    tight in the worst case.
  \end{question}

  \begin{question}
    Let $P$ be a set of $n$ points in $\R^2$, and let
    $\mathit{NN}(p)$ denote the (Euclidean) \emph{nearest-neighbor} of
    $p$. That is, $\mathit{NN}(p) = \argmin_{q \in P} \|pq\|$.

    \begin{subquestions}
      \subquestion[10] Prove that the Voronoi regions of $p$ and $\mathit{NN}(p)$ are
      adjacent in the Voronoi diagram $\mathit{VD}(P)$ of $P$.

      \subquestion[10] Describe an $O(n \log n)$ time algorithm to compute, for every
      point $p \in P$, its nearest neighbor $\mathit{NN}(p)$.
    \end{subquestions}
  \end{question}

  \begin{question}[15]
    Let $R$ be a set of $n$ ``red'' points in $\R^2$, let $B$ be a set
    of $n$ ``blue'' points in $\R^2$, let
    $\mathit{age} : R \cup B \to \R$ be a function that assigns an age
    to every point, and let $\Delta > 0$ be some real number. You can
    again assume that all coordinates and ages are unique, and that
    there are no three colinear or four corcicular points. Design an
    algorithm to compute, for every red point $r \in R$, the closest
    (in terms of the Euclidean distance) blue point $b$ among $B$ for
    which
    $\mathit{age}(b) \in [\mathit{age}(r), \mathit{age}(r) +
    \Delta]$. Your algorithm should run in subquadratic time.
  \end{question}


\end{document}
