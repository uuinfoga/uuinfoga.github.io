\documentclass{article}
\usepackage{xspace}
\usepackage{fancyhdr}
\usepackage{graphicx}
\usepackage{amsmath,amssymb,amsthm}
\usepackage{microtype}
\usepackage{enumitem}
\usepackage{ifthen}
\usepackage{totcount}
\usepackage[usenames]{xcolor}
\usepackage[most]{tcolorbox}
\usepackage[a4paper, margin=1in]{geometry}


\graphicspath{ {./figures/} }

%--------------------------------------------------------------------------------

\newcommand{\edition}{2024-2025}
\newcommand{\deadline}{22 November 2024, 09:00}
\newcommand{\hwNum}{1}

%--------------------------------------------------------------------------------
% Style stuff
\pagestyle{fancy}
\lhead{Geometric Algorithms (INFOGA \edition)}
\chead{}
\rhead{\mytitle}
\lfoot{INFOGA \edition}
\cfoot{}
\rfoot{\thepage}
\renewcommand{\headrulewidth}{0.4pt}
\renewcommand{\footrulewidth}{0.4pt}
\renewcommand{\familydefault}{\sfdefault}

%--------------------------------------------------------------------------------
% Question macro's

\newcommand{\withpoints}[1]{%
  \addtocounter{pointscounter}{#1} \printpoints{#1}
}
\newcommand{\printpoints}[1]{%
   \ifthenelse{#1 = 0}
              {}
              {\textit{(#1 points)}}\mbox{}
}



\newenvironment{question}[1][0]{\begin{tcolorbox}[parbox=false
                                      ,breakable=true
                                      ,enhanced jigsaw
                                      ,title=\bfseries Question
                                      \stepcounter{questionscounter}\arabic{questionscounter}
                                      \withpoints{#1}]
                         }
                         {\end{tcolorbox}}
\newenvironment{subquestions}{\begin{enumerate}[leftmargin=*
                                        ]}
                             {\end{enumerate}}
\newcommand{\subquestion}[1][0]{\item \withpoints{#1}}

\newtotcounter{questionscounter}
\newtotcounter{pointscounter}

\newcommand{\numquestions}{\total{questionscounter}}
\newcommand{\numpoints}{\total{pointscounter}}

%--------------------------------------------------------------------------------
\newcommand{\myremark}[3]{\textcolor{blue}{\textsc{#1 #2:}} \textcolor{red}{\textsf{#3}}}
% \renewcommand{\myremark}[3]{}
\newcommand{\frank}[2][says]{\myremark{Frank}{#1}{#2}}

%--------------------------------------------------------------------------------
% Theorem Environments
\newtheorem{theorem} {Theorem}
\newtheorem{lemma}[theorem] {Lemma}
\newtheorem{corollary}[theorem] {Corollary}
\newtheorem{problem}[theorem] {Problem}
\newtheorem{observation}[theorem] {Observation}
\newtheorem{claim}[theorem] {Claim}
\newtheorem{invariant}[theorem] {Invariant}

%--------------------------------------------------------------------------------
% Otherwise useful Macro's
\newcommand{\eps}{\ensuremath{\varepsilon}\xspace}
\DeclareMathOperator{\argmin}{argmin}

\newcommand{\mkmbb}[1]{\ensuremath{\mathbb{#1}}\xspace}
\newcommand{\R}{\mkmbb{R}}

\newcommand{\mkmcal}[1]{\ensuremath{\mathcal{#1}}\xspace}
\newcommand{\RR}{\mkmcal{R}}

%--------------------------------------------------------------------------------
% The Actual document

\newcommand{\mytitle}{Homework Exam \hwNum\xspace\edition}
\title{\mytitle}
\date{\textbf{Deadline: } \deadline}
\author{{\color{red} My name and StudentID go here!}}


\begin{document}
\maketitle
\thispagestyle{fancy}
  \noindent
  This homework exam has \numquestions\xspace question for a total of
  \numpoints\xspace points. You can earn an additional point by a careful
  preparation of your hand-in: using a good layout, good spelling, good
  figures, no sloppy notation, no statements like ``The algorithm runs in
  $n\log n$.''  (forgetting the $O(..)$ and forgetting to say that it concerns
  time), etc. Use lemmas, theorems, and figures where appropriate.

  \begin{question}
    Let $P$ be a set of $n$ points in $\R^2$, and let \RR be
    a set of $m$, possibly pairwise intersecting, polygonal
    regions. You can assume by general position that all points and
    vertices have unique coordinates, and that no three points and
    vertices are colinear. The depth $d_\RR(p)$ of a point $p$ with
    respect to \RR is the number of regions from \RR that contain
    it.

    \begin{subquestions}
      \subquestion[6] Consider the scenario in which all regions in
      \RR are axis-aligned rectangles. Design an $O((n+m)\log (n+m))$
      time algorithm that computes, for each point in $P$, its depth
      with respect to \RR. Prove that your algorithm is correct and
      achieves the desired running time.
      \subquestion[2] An equilateral triangle is axis aligned when one of
      its sides is axis aligned. Extend your algorithm from question
      (a) to the case where all regions in \RR are axis aligned
      equilateral triangles. Briefly argue that your algorithm is
      correct and analyze its running time.

      (Note that you do not have to repeat the entire description from
      question (a). Focus on what is different.)

      \subquestion[1] Can your algorithm (from question (a) and/or (b))
      still compute the depths in $O((n+m)\log (n+m))$ time in case
      the regions in \RR are arbitrary triangles? Briefly argue
      why/why not.
    \end{subquestions}

  \end{question}
\end{document}
